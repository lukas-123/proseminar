\documentclass[hidelinks]{article}

%% Deutsche Silbentrennung und Sprache (neue Rechtschreibung)
\usepackage[ngerman]{babel}
%% Verwende Umlaute direkt
\usepackage[utf8x]{inputenc}
%% Hyperlinks für interne Referenzen
\usepackage{hyperref}
%% Grafiken einbinden
\usepackage{graphicx}
%% Paket für Unterabbildungen pro Abbildung
%\usepackage{subfig}
\usepackage{amsmath}
\usepackage{amssymb}

\usepackage[margin=1.5in]{geometry}
\usepackage{cite}

% Titel der Arbeit
\title{Elliptische Kurven}
% Angaben zum Author
\author{Kevin Kappelmann, Lukas Stevens}
\pagestyle{plain}

%------------------------------------------------------------------------------
\begin{document}

\maketitle
\newpage
\tableofcontents
\newpage

Darstellungsformen nicht vergessen! Edwards Kurven und so

Beispiel zitat
\cite[chapter, p.~215]{blablubb}

\section{Motivation und Geschichte}
Macht Kevin\\
Einleitung, warum elliptische Kurven, etc. (geschichtliches?)
\section{Grundbegriffe}
\subsection{Affine Ebenen}
Definition, Beispiele
\subsection{Projektive Ebenen}
Definition
\subsubsection{Die projektive Ebene PG(2, $\mathbb{F}$)}
Konstruktion, Beispiel
\subsubsection{Konstruktion affiner Ebenen aus projektiven Ebenen}
Beweis, Beispiel
\section{Elliptische Kurven $E$}
Macht Lukas\\
\subsection{Definiton elliptischer Kurven}
Weierstraßgleichung, Nullstellenmenge des Polynoms, Charakteristiken(Singularitäten), affine Koordinatentransformation?
\subsection{Die unendliche Gerade über PG(2, $\mathbb{F}$)}
Isomorphismus von $\mathbb{F}^2 \rightarrow \mathcal{P}_U$
\subsection{Affine Darstellung elliptischer Kurven}
Erklärung, Beispiel(Graphen)
\section{Eine Gruppe über $E$}
Macht Kevin bis 4.3\\
\subsection{Tangenten elliptischer Kurven}
\subsection{Schnittpunkte von Geraden mit elliptischen Kurven}
Unendlich ferne Gerade mit Schnittpunkt $\mathcal{O}$, Affine Geraden, Parallele zur y-Achse
\subsection{Die Schnittpunkt-Verknüpfung $\oplus $ über $E$}
Definition, Beweis der Abgeschlossenheit, graphische Interpretation
\subsection{Die Gruppe $(E, +)$}
Macht Lukas bis fertig\\
Gruppe ist abelsch mit neutralem Element $\mathcal{O}$, Beispiel
\section{Anwendung elliptischer Kurven in der Kryptologie}
\subsection{Elgamal}
Welche Charakteristiken für elliptische Kurven, Domänenparameter
\subsection{Noch einen für Signaturen}
Welche Charakteristiken für elliptische Kurven, Domänenparameter

\nocite{*}
\bibliographystyle{plain}
\bibliography{quellen}
\end{document}
