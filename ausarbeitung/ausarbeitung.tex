\documentclass[hidelinks]{article}

%% Deutsche Silbentrennung und Sprache (neue Rechtschreibung)
\usepackage[ngerman]{babel}
%% Verwende Umlaute direkt
\usepackage[utf8x]{inputenc}
%% Hyperlinks für interne Referenzen
\usepackage{hyperref}
%% Grafiken einbinden
\usepackage{graphicx}
%% Paket für Unterabbildungen pro Abbildung
%\usepackage{subfig}
\usepackage{enumerate}
\usepackage{amsmath}
\usepackage{amssymb}
% Proof system
\usepackage{amsthm}
\newtheorem{thm}{Satz}[section]
\newtheorem{lem}[thm]{Lemma}
\newtheorem{defn}[thm]{Definition}
\usepackage{xpatch}
\makeatletter
\xpatchcmd{\thm}{\thmheadpunct{.}}{\thmheadpunct{\\}}{}{}
\makeatother

% Seitenränder
\usepackage[margin=1.5in]{geometry}
% Zitate
\usepackage{cite}
% Tabellen
\usepackage[table]{xcolor}
\definecolor{lightblue}{HTML}{8ADAF2}
\definecolor{lightred}{HTML}{F2A28A}
\definecolor{lightgreen}{HTML}{A6F28A}

\setlength{\parindent}{0pt}

\newcommand{\pgtwo}{PG(2, $\mathbb{F}$)\ }

\addto{\captionsngerman}{\renewcommand{\refname}{Literaturverzeichnis}}

% Titel der Arbeit
\title{Elliptische-Kurven-Kryptographie}
% Angaben zum Author
\author{Kevin Kappelmann, Lukas Stevens}
\pagestyle{plain}

%------------------------------------------------------------------------------
\begin{document}

\pagenumbering{gobble}
\maketitle
\newpage
\tableofcontents % Inhaltsverzeichnis
\listoffigures % Abbildungsverzeichnis
\listoftables % Tabellenverzeichnis 
\newpage

\pagenumbering{arabic}

\begin{sloppypar}
Darstellungsformen nicht vergessen! Edwards Kurven und so

\section{Einleitung und Motivation}
Kryptosysteme wie RSA, Diffie-Hellman\footnote{In der jeweiligen Implementierung als Gruppe über ganze Zahlen} und ElGamal\footnotemark[\value{footnote}], die sich auf die Schwere der Primafaktorzerlegung bzw.\ dem diskreten Logarithmenproblem über Ganzzahlen stützen, benötigen sehr große Schlüssellängen, um eine ausreichend hohe Sicherheit zu garantieren. 
Daraus ergibt sich sowohl eine hoher Energie- als auch Speicherbedarf für die Berechnung der Algorithmen, was vor allem für Microchips und eingebettete Systeme ein Problem darstellt.\\
Eine Lösung für dieses Problem sind elliptische Kurven. Diese algebraischen Kurven tragen eine Gruppenstruktur, über die das diskrete Logarithmenproblem deutlich schwerer lösbar ist, als über Gruppen mit Ganzzahlen.
Kryptosysteme, die auf elliptische Kurven beruhen, kommen dadurch mit erheblich kürzeren Schlüsseln bei vergleichbarer Sicherheit aus.\cite[Seite~53]{nist}\\
Nachfolgende Tabelle verdeutlicht diesen Sachverhalt. Spalte 1 kennzeichnet die maximale Sicherheit (in Bits) für den jeweiligen Algorithmus und der angegebenen Schlüssellänge (in Bits). Rot markierte Felder gelten als kryptographisch unsicher, grüne als aktuell sicher.

\begin{table}[h]
\centering
	\begin{tabular}{| c | c | c |}
	\hline
	\rowcolor{lightblue}
	Sicherheitsniveau & RSA/Diffie-Hellman\footnotemark[\value{footnote}] & Elliptische-Kurven\\ \hline
	\rowcolor{lightred}
	$\le80$ 	& 1024 & 160-223 \\ \hline
	\rowcolor{lightgreen}
	112 	& 2048 & 224-255 \\ \hline
	\rowcolor{lightgreen}
	128 	& 3072 & 256-383 \\ \hline
	\rowcolor{lightgreen}
	192 	& 7680 & 384-511 \\ \hline
	\rowcolor{lightgreen}
	256 	& 15360 & 512+ \\ \hline
	\end{tabular}
\caption{Vergleich Schlüssellängen}
\end{table}

Die Verwendung elliptischer Kurven in der Kryptographie wurde Mitte der 1980er Jahre von Neal Koblitz\cite{koblitz} und Victor S. Miller\cite{miller} unabhängig voneinander vorgeschlagen. Aufgrund der vorteilhaften Eigenschaften gewinnt die Elliptische-Kurven-Kryptographie (kurz ECC für Elliptic Curves Cryptography) stets mehr an Bedeutung und löst ältere Verfahren wie RSA in den verschiedensten Bereichen ab. Vor allem in Umgebungen mit begrenzten Kapazitäten, wie z.B.\ Smartcards, ist ECC bereits weit verbreitet.\\
So verwendet beispielsweise Österreich seit 2004 als Vorreiter für alle gängigen Bürgerkarten ECC\cite{austria}. Aber auch die Reisepässe der meisten Europäischen Staaten nutzen inzwischen meist in einer Form ECC.\cite{eu}


\section{Grundbegriffe}
Um elliptische Kurven einführen zu können, müssen wir uns zunächst mit affiner und projektiver Geometrie auseinander setzen. Wir führen hierfür zunächst allgemein die Begriffe der affinen und projektiven Ebene ein und konstruieren uns eine projektive Ebene \pgtwo über einen beliebigen Körper $\mathbb{F}$.

\subsection{Affine Ebenen}
\begin{defn} Es sei $\mathcal{A}$ eine Menge und $\mathcal{G}$ eine Teilmenge der Potenzmenge von $\mathcal{A}$, d.h. $\mathcal{G}\subseteq Pot(\mathcal{A})$.
Das Paar $(\mathcal{A},\mathcal{G})$ heißt affine Ebene, falls folgende vier Bedingungen erfüllt sind:
	\begin{enumerate}
		\item[(A1)] $\forall G\in\mathcal{G}:|G|\ge 2$ (auf jeder Gerade liegen mindestens zwei Punkte).
		\item[(A2)] Zu je zwei Elementen $a, b\in \mathcal{A}$ mit $a\ne b$ existiert genau ein $G\in\mathcal{G}$ mit $a, b \in G$ (durch zwei verschiedene Punkte geht genau eine Gerade).\\
		Wir schreiben $\overline{a,b}$ für dieses G.
		\item[(A3)] Zu $G\in\mathcal{G}$ und $a\in\mathcal{A}\setminus G$ existiert genau ein $G'\in\mathcal{G}$ mit $a\in G'$ und $G\cap G'=\emptyset$ (durch jeden Punkt geht genau eine Gerade, die zu einer gegebenen Gerade parallel ist).
		\item[(A4)] Es gibt drei Punkte $a,b,c\in\mathcal{A}$ mit $c\notin\overline{a,b}$ (es gibt drei Punkte, die nicht alle auf einer Gerade liegen).
	\end{enumerate}
\end{defn}
Die Menge $\mathcal{A}$ nennt man die Punktmenge und die Menge $\mathcal{G}$ die Geradenmenge der affinen Ebene $(\mathcal{A},\mathcal{G})$.\\\\
TODO
Anschaulich beschreibt eine affine Ebene den uns bekannten geometrischen Raum

\subsection{Projektive Ebenen}
Definition
\subsubsection{Die projektive Ebene \pgtwo}
Konstruktion, Beispiel
\subsubsection{Konstruktion affiner Ebenen aus projektiven Ebenen}
Beweis, Beispiel
\section{Elliptische Kurven $E$}
Macht Lukas\\
\subsection{Definiton elliptischer Kurven}
%Weierstraßgleichung, Nullstellenmenge des Polynoms, Charakteristiken(Singularitäten), affine Koordinatentransformation?
Wir haben bereits die projektive Ebene \pgtwo über beliebige Körper $\mathbb{F}$ eingeführt.\\
Diese hat die folgende Punktemenge:
\begin{equation*}
    P = \left\{(u:v:w) \mid (u,v,w) \in \mathbb{F}^3 \setminus (0,0,0) \right\}
\end{equation*}
Nun wollen wir die Punktemenge $E$ der elliptischen Kurve einführen. Dazu benötigen wir Polynome in drei Unbekannten.
Der Polynomring mit drei Unbekannten über $\mathbb{F}$ ist mit 
\begin{equation*}
    \mathbb{F}[X,Y,Z] = \left\{ \sum_{k,l,m \geq 0} a_{k,l,m} \thinspace X^k Y^l Z^m \mid a_{k,l,m} \in \mathbb{F} \right\}
\end{equation*}
definiert. 
$F(X,Y,Z) = \sum_{k,l,m \geq 0} a_{k,l,m} \thinspace X^k Y^l Z^m \in \mathbb{F}[X,Y,Z]$ wird Polynom genannt. \\ 
\newline
Eine elliptische Kurve $E$ ist durch die Lösung der Weierstraß-Gleichung 
\begin{equation*}
    Y^2Z + a_1XYZ + a_3YZ^2 = X^3 + a_2X^2Z + a_4XZ^2 + a_6Z^3
\end{equation*}
gegeben, wobei gilt $a_i \in \mathbb{F}$. Da der zugrundeliegende Raum \pgtwo eine projektive Ebene ist, handelt es sich um eine projektive Kurve. 
Wenn man die Gleichung als Polynom 
\begin{equation*}
    F(X,Y,Z) = Y^2Z + a_1XYZ + a_3YZ^2 - X^3 - a_2X^2Z - a_4XZ^2 -a_6Z^3
\end{equation*}
schreibt, dann ist $E$ genau die Nullstellenmenge des Polynoms $F$.
\\ TODO char etc.

\subsection{Die unendliche Gerade über \pgtwo}
%Isomorphismus von $\mathbb{F}^2 \rightarrow \mathcal{P}_U$
Um in~\ref{affine-darstellung} eine affine Darstellung herzuleiten, müssen wir $(\mathcal{P,G}) =$ \pgtwo nochmal betrachten.
Wir wählen dazu eine Gerade $U \in \mathcal{G}$ aus. Prinzipiell kann dazu jede Gerade gewählt werden. Es ist jedoch von Vorteil eine bestimmte Gerade zu wählen um das Rechnen mit der Weierstraßgleichung zu vereinfachen. \\
Dazu wählen wir die Verbindungsgerade $U = \overline{P,Q}$ der Punkte $P = (1:0:0)$ und $Q = (0:1:0)$, d.h. $U = \left\{ (x:y:z) \in \mathcal{P} \mid z = 0 \right\}$. Diese Menge $U$ bezeichnen wir im Folgenden als unendlich ferne Gerade. 
\\ TODO Lemma
\subsection{Affine Darstellung elliptischer Kurven} \label{affine-darstellung}
Erklärung, Beispiel(Graphen)
\section{Eine Gruppe über $E$}
Macht Kevin bis 4.3\\
\subsection{Tangenten elliptischer Kurven}
\subsection{Schnittpunkte von Geraden mit elliptischen Kurven}
Unendlich ferne Gerade mit Schnittpunkt $\mathcal{O}$, Affine Geraden, Parallele zur y-Achse
\subsection{Die Schnittpunkt-Verknüpfung $\oplus $ über $E$}
Definition, Beweis der Abgeschlossenheit, graphische Interpretation
\subsection{Die Gruppe $(E, +)$}
Macht Lukas bis fertig\\
Gruppe ist abelsch mit neutralem Element $\mathcal{O}$, Beispiel
\section{Anwendung elliptischer Kurven in der Kryptologie}
\subsection{ElGamal}
Welche Charakteristiken für elliptische Kurven, Domänenparameter
\subsection{Noch einen für Signaturen}
Welche Charakteristiken für elliptische Kurven, Domänenparameter

\end{sloppypar}
\newpage
\nocite{*}
\bibliographystyle{plain}
\bibliography{quellen}
\end{document}
